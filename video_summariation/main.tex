\documentclass{beamer}

% Theme choice (feel free to explore others)
\usetheme{Madrid} 
\usecolortheme{dolphin}

\title[AI Engineer World's Fair 2024]{Open Models Track}
\subtitle{A Summary}
\author{Generated by Gemini 1.5 Pro}
\date{\today} 

\begin{document}

\frame{\titlepage}

\section{Introduction}

\begin{frame}{Setting the Stage}
  \begin{itemize}
    \item The Open Models track at AI Engineer World's Fair 2024 highlighted the rapid progress and exciting potential of open-source Large Language Models (LLMs).
    \item Researchers and developers from leading organizations showcased advancements in training, applications, and fostering a collaborative open-source AI community.
    \item This presentation summarizes the key takeaways and future directions discussed during this insightful event. 
  \end{itemize}
\end{frame}

\section{Key Sections}

\subsection{Meta AI: Democratizing Access}

\begin{frame}{Meta AI: Democratizing Access}
  \begin{itemize}
    \item Focus on making advanced LLMs widely accessible.
    \item In-depth explanation of the three LLM training stages:
      \begin{itemize}
        \item Pre-training
        \item Instruction Tuning
        \item Learning from Human Feedback
      \end{itemize}
    \item Introduction of Code Llama: open-source, code-generating LLM available on multiple platforms.
  \end{itemize}
\end{frame}

\subsection{Txt: Structure Generation}

\begin{frame}{Txt: Taming LLMs with Structure}
  \begin{itemize}
    \item Addressed the challenge of consistent structured output from LLMs.
    \item Introduced Outlines: Python library for structure generation (JSON, regular expressions, etc.).
    \item Benefits: 
      \begin{itemize}
        \item Reliable structured data generation.
        \item Minimal inference overhead (sometimes even faster).
        \item Improved sample efficiency.
        \item Enhanced performance of open-source models. 
      \end{itemize}
  \end{itemize}
\end{frame}

\subsection{Cohere: Developer-First RAG}

\begin{frame}{Cohere: Developer-First RAG}
  \begin{itemize}
    \item Focused on building developer-friendly tools for Retrieval Augmented Generation (RAG).
    \item Discussed challenges in RAG system development.
    \item Showcased Command R model family optimized for RAG, highlighting performance and cost-effectiveness.
    \item Open-sourced their UI toolkit to empower developers in building RAG applications. 
  \end{itemize}
\end{frame}

\subsection{UNSloth: Debugging LLMs}

\begin{frame}{UNSloth: Debugging and Optimization}
  \begin{itemize}
    \item Highlighted the importance of bug identification and fixing in open-source LLMs (focus on Llama 3).
    \item Detailed common fine-tuning pitfalls and the need for careful configuration and pre-processing.
    \item Presented UNSloth's open-source platform:
      \begin{itemize}
        \item Pre-built bug fixes.
        \item Automatic chat template generation.
        \item Support for long-context fine-tuning. 
      \end{itemize}
  \end{itemize}
\end{frame}

\subsection{Liquid AI: Model Merging}

\begin{frame}{Liquid AI: Model Merging}
  \begin{itemize}
    \item Provided an overview of fine-tuning techniques and strategic application for LLM performance. 
    \item Discussed when to choose fine-tuning over prompt engineering.
    \item Explained various fine-tuning methods (full, LoRA, 4-bit quantization).
    \item Introduced Model Merging as a powerful technique for combining strengths of different fine-tuned models.
    \item Showcased merging techniques (Slurp, DeLoRA, Pass-through, Mixture of Experts) and real-world examples.
  \end{itemize}
\end{frame}

\section{Conclusion}

\begin{frame}{The Future of Open LLMs} 
  \begin{itemize}
    \item The Open Models track offered an inspiring look at the future of LLM development.
    \item It highlighted a collaborative ecosystem focused on groundbreaking research, accessibility, and responsible AI deployment.
    \item Open-source initiatives play a crucial role in shaping the future of LLMs, making them more powerful and beneficial for all. 
  \end{itemize}
\end{frame}

\end{document}